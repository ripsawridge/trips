\startcomponent stage5
\product quicksilver-book
\project quicksilver

\chapter{Stage Five}

South of Bivio, we approach and cross the Septimer Pass (aka pass da Sett),
which marks the southern terminus of the Albula Alps, our range since the middle
of Stage Three.

From the alpine resupply center of Bivio, we'll walk past the entrace to the
broad Valletta Valley, and turn south into it's neighbor the Tgavretga. The
Septimerpass (2310 m) is marked by some hidden alpine lakes 50 meters above the
pass on the west. A quick descent to the south reaches trail in the Val Marox at
1799 meters. Climbing west through beautiful country into the Val da la Duana
marked by alpine lakes, we'll exit this faerie tale land by the Pass da la Duana
(2694 m). A steep and dramatic descent to relatively low country of the great
Val Bregagli follows. There are many trails here, but it seems wisest to descend
to Soglio (1090 m), and then perhaps down to Castasegna on the river at 756
meters.

Our long romp through high alpine country is over, and it's only the whimsical
desire to see the Como Lake first from an echoing high country that lifts us
up again. A gentle walk along the river in the valley floor past numerous
villages leads to the metropolis of Chiavenna, it's sister city on the west side
of the Liro, San Mamete, and then San Vittore and Gordona. Supplies for the last
high jaunt should be obtained in these bustling city states.

From Gordona, a road switchbacks up the western valley wall, then enters the
broad Valle Bodengo. The settlement of Bodengo (1030 m) may provide
lodging. Beyond this point a long alpine climb to the Correggia Pass (2201 m)
awaits, marked by narrowing valley walls and peaks with many granitic
crags. Ideally a scramble up one of the peaks of the pass could be made, perhaps
to the Pizzo San Pio (2304 m), or the Pizzo Campanile (2458 m), a peak with the
additional requirement of crossing the tiny Pass del Valon.

Descent from the Correggia appears steep, and trail may be faint. The Lago di
Darengo lurks below in a kind of Sarlac Pitt at 1781 meters, and is guarded by
the Rifugio di Darengo (1790 m) on the southeastern shore. Our trail follows the
outlet stream down to Alpe Darengo at 1378 meters, soon passing the Rifugio
Pianezza at 1252 meters, and then curving to the south after the village of
Borgo, and then Baggio (930 m), descends to the shrine of the Madonna di Livo
(659 m). An uneventful road walk to the south reaches Livo.

From here, a short walk through Peglio on the road and a short-cut trail to San
Carlo (334 m), brings the pilgrim with aching feet to Gravedona. At 202 meters
above the sea, with the expanse of the Como Lake all around, the journey is
over.


\trfigure{}{West from the Septimer Pass.}{s5_1.pdf}
\trfigure{}{Down to the valley.}{s5_2.pdf}
\trfigure{}{On to Chiavenna.}{s5_3.pdf}
\trfigure{}{Up from Gordona.}{s5_4.pdf}
\trfigure{}{Over and down to Gravedona.}{s5_5.pdf}

\stopcomponent
