\startcomponent introduction
\product quicksilver-book
\project quicksilver

\chapter{Introduction}

I wanted to walk a long ways. And my line across the mountains should be
diagonal, to express my love for the hills. I don't seek the straightest way
through the domain of monsters. I wanted to start my adventure at my door, with
pack already on my back. The best adventures begin from home and lead into
unknown. I didn't want mitigations. I didn't want compromises, as in when
someone claims to walk from New York to Los Angeles, but when questioned later
after the slide show says, ``well, I actually began in Newark, of course.''

I didn't want to limp into my destination. The destination, although somewhat
arbitrary, demands respect. A vast, hot, nearly sea-level lake in Italy, Lake
Como could be spoiled by a too-direct approach. I've never been there, but I’ve
seen the Garda Lake to the east appear ugly down among the broiling streets and
construction sites. I would hate to see the shimmering lake through a patina of
orange cones and sweaty families, milling about the shore feeling vaguely
cheated, as they do.

No, I needed the lake to be a holy place. I would end at Gravedona, because it
sat on the northern shore with a long view straight south. The name, Gravedona,
sounds good to me. A beautiful woman made more what she is through sorrow, I
thought, having no real idea of the meaning of the name or the history of the
town. And I must come down to the lake, down from the highest pass. My first
view of the lake would be from the Passo Correggia, more than a vertical mile
above. If I have the energy I'll climb the Pizzo Campanile beside the pass, for
a last look back to the Northeast, my home.

And what about the miles in between the good start and the happy end? I had an
idea of what I wanted there too. I wanted all the other things I love in
mountains. Good, Leave-it-to-Beaver exertion in the sun, with those moments of
simple pride looking back at a country traversed. ``Whelp!'' I would be saying at
these times. ``Early birds and worms and such in an old cycle!''

I wanted a hint of the radical as well. The Unknown, the Unwise, the fever-dream
of the dreamer hideously awake and putting on his boots in the pre-dawn! I would
include some peaks along the way to scramble up and over with barely passable
equipment. Because these dreamers deepen their journeys immensely with their
foolishness. A red wine suffuses a full day spent in madness, in love. The mind
bifurcates, dealing on the one side with practicalities...mustn't slip here! On
the other side, the meaning of life itself is put on the table and examined
through action. The questions don't have answers, but they are twisted this way
and that, grappled with in a way that cannot be explained to the armchair
mountaineer. Such occasional reality baths would scrub my soul, as they have
before.

I wanted the capricious. I love to improvise, and the opportunity to toy with
the radical idea has entangled many ``evenings before'' with living sprites of
mystery. ``How about if I go over the peak?'' is a good question to be confronted
with. But so is ``how about if I sleep in the meadow all day and jog around this
peak grouping before dinner?'' I would need to leave room for possibility.

I am both leery and desirous of people. My idea is that they come like water,
faster and more along known, popular and named routes, but like trickles on a
November morning when off to the side on queer ways. I would enter and exit
their streams in a way where the change was always refreshing. I don't like to
soak in the habitual misanthropy that a love of wilderness seems to imply. If I
sit too long in that particular tub I begin to feel counterfeit. That is, who am
I to demand vast space for me and me alone: my impulse should be to share
instead. At the same time, crowds are certainly not what I want!

I want the ability to make daylong emotional journeys aided by the steady
metronome of gait. The best terrain for such inner meditations is given by long
traverses, ``balcony'' trails, of which the Alps have many. Because of my need to
travel in one direction, I don't actually get many of these, and I'll have to
settle for occasional valley walks to provide more gentle terrain.


\stopcomponent
