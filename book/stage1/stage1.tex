\startcomponent stage1
\product quicksilver-book
\project quicksilver

\chapter{Stage One}

\startcolumns[n=2, rule=on]

It seems most logical to begin at the Fernpaß (1216 m). We walk west, north (!)
then west again into the Kälbertal, climbing to near the head of this valley
dominated by the Roter Stein (2366 m). At the Galtberghütte (1790 m) we turn
south, escaping the valley for a climb of the Ostliches Kreuzjoch (2231 m). We
make south from the summit at a similar elevation for the Loreascharte (2315 m)
and drop down to a junction called Am Kragen then a bend north along the
Loreggbach on a forest trail that bends around a buttress past the abandoned
Bichlwaldhütte and into the deep bed of the Rotlech river, traveling south
towards the broad, flat Tarrentonalm at 1519 meters.


Our first major climb group behind us, we ascend southwest up a valley below the
rounded Hinterberg peak and cross the Hinterbergjoch (2202 m) at the valley
head. After the slightly lower Kromsattel we arrive at the Anhalter Hütte and
the Kromsee, then turn into the Steintal for a short climb to the Steinjöchl
(2198 m) followed by a traversing descent to the Hahntennjoch (1894 m). A gentle
walk west on a trail along the road down to Boden (1356 m) ends this mountain
group.

From Boden, we have a long walk due south up the Angerlebach to the Hanauer
Hütte (1922 m) and the little pass by the Dremelspitze, whose 2733 meters offers
a superb scrambling climb of about 300 meters elevation gain. Down to the
Steinsee and the Steinsee Hptte (2061 m) then back up and following a westerly
traversing trail to the Gebäudjoch (2452 m) and down to the Württemberger Haus
(2220 m). An alternate approach to this point from the Hanauer Hütte could be
made by the pass north of the Parzinnspitze (~2355 m), down into the Hintere
Gufelalpe, back up by the Bittrichsee and over an unnamed pass (~2545 m) and
directly down to the house. From the Württemberger Haus, continuing west and a
little south up and over the Groß bergkopf (2612 m) and the Großbergjoch just
below we make our way to the Seescharte, then drop through some beautiful lakes
to the Memminger Hütte (2242 m).

A deep drop into the valley of the Zammer Parseier at around 1700 meters, then
immediately back up (somewhat unpleasantly, perhaps!) the Longkar under the Rote
Platte (2831 m) to the Winterjoch at 2528 meters. From here we descend southwest
to the Ansbacher Hütte (2376 m), dropping steeply to the Fritzhütte (1727 m)
before reaching the town of Schnann, at 1167 meters, deep in the trunk valley
with St. Anton am Arlberg just a few kilometers to the west.

This ends Stage One, by any sane reckoning. It would be nice to include the
Parseierspitze (3036 m), a fantastic view peak of the area, but it seems likely
that a southern descent would bring you to low elevations in the area of
Imst. The descent-and-reascent in the Zammer Parseier is unfortunate, and if a
way to connect to the Parseierscharte at the head of this valley could be found,
it would probably be preferred over the trip to the Winterjoch.
\stopcolumns

\trfigure{}{Start at the Fernpass.}{s1_1.pdf}
\trfigure{}{The Schweinsteinjoch.}{s1_2.pdf}
\trfigure{}{The Hahntennjoch.}{s1_3.pdf}
\trfigure{}{The Steinsee Hütte.}{s1_4.pdf}
\trfigure{}{Near the Parseierspitze.}{s1_5.pdf}
\trfigure{}{Approaching the Inn.}{s1_6.pdf}

\stopcomponent
