\startcomponent stage4
\product quicksilver-book
\project quicksilver

\chapter{Stage Four}

Now in the southern terminus of the Dürr Valley, dominated by the Scalettahorn
(3068 m), we climb south to the Seeböden, a lake a 2335 meters. Our path climbs
screes well above the lake through alpine country to the Scalettapass (2606 m).
We proceed over and down the other side to the Alp Funtauna on a broad valley
floor at 2192 meters. 

A relaxing walk slowly up the valley of the Funtauna eventually reaches the
Chamanna Kesch (2630 m), below the imposing and glacier-clad Piz Kesch (3417
m). We have two possibilities here, one is to climb the Porchabella Glacier to
the Porta d’Es-cha (3008 m), for a quick descent to the Chamanna d’Es-cha, and
then to the Albulapass. This is attractive, both for the views and the
efficiency, but the Porchabella Glacier is significantly crevassed, and the ease
of access to rocks of the Porta d’Es-cha is uncertain. It could stress the
limits of lightweight equipment (i.e., aluminum crampons for tennis shoes), and
the reasonable risk threshold for a solo traveler on a glacier. The alternative
is to descend all the way to Bergün (1367 m), and follow a combination of trail
and road up to the point near the Albulapass where our route continues. I would
suggest that it’s more likely that this option be chosen, unless strong
assurances regarding the state of the glacier can be found. What’s more, owing
to the unusual nature of your lightweight equipment and esoteric (though
comprehensive) background regarding alpine travel, it would be hard to find
someone with experience of the route over the Porta d’Es-cha to judge your
ability to overcome it. Thus far in the description of stages, this is the
riskiest proposition yet.

Assuming the Porta d'Es-cha could be traversed, a steep alpine descent is made
posthaste to the Chamanna d'Es-cha (2594 m). Easy travel from here along a
traversing trail to the south reaches the Fourcla Gualdauna (2494 m), and an
uneventful descent to the road a little east of the Albulapass (2312 m). It
appears that the road can be walked up to the pass, or a trail on the south side
of the road which would be more pleasant for the feet.

Descend a little ways on the west side of the path, then when the valley walls
broaden slightly, leave the road on trail to the south for the scenic climb to
the Fuorcla Crap Alv (2466 m). It's necessary to make a traversing descent to
the east on the south side of the pass, arriving at a junction of river and road
at 2018 meters. Now a long journey up valley to the west follows, making for the
Chamanna Jenatsch (2652 m) where sore bodies should find a welcome bed.

An exciting alpine day awaits, and an early start is advised. Leave the hut and
travel due south making for an unnamed lake below the Piz d'Agnel at 2766
meters, then climbing up faint trail and south through the Fuorcla d'Agnel at
2986 meters. The descent from the pass is marked by small lakes and then a
junction with trail heading up to the Fuorcla Leget at about 2530 meters. Take
this junction, climbing up to the pass at 2715 meters, then heading west,
passing a lake then descending steeply into the Val da Natons. Follow the valley
to the Alp Natons just above the treeline, admiring the gorgeous view of the
Marmorera Lake. A choice of high or low trails depending on available energy
leads to the relatively large settlement of Bivio.

\trfigure{}{The Scaletta Pass.}{s4_1.pdf}
\trfigure{}{Fourcla Crap Alv.}{s4_2.pdf}
\trfigure{}{Chamanna Jenatsch.}{s4_3.pdf}
\trfigure{}{The Septimer Pass.}{s4_4.pdf}

\stopcomponent
