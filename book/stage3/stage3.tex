\startcomponent stage3
\product quicksilver-book
\project quicksilver

\chapter{Stage Three}

Into Switzerland.

From the road, pass and houses at the Zeinisjoch we walk around either side of
the Stausee Kops lake (the left side is a dam), and climb gently up to a fine
pass view from the Breitspitze (2196 m), a suitable entry to the Silvretta
Alps. A descent to the valley floor of the Vallula to around 1700 meters
follows, then a climb to an eastern exit before reaching the upper valley to
gain a pass I call “P2515;” it resides just north of the Kleine Vallüla summit.
From here a long, gradually traversing descent to the Silvretta Haus at the
Biehlerhöhe (~2030 m) is accomplished on a meadowy southern alp called the
Maißboden. Possibly, Swiss Francs could be acquired here. There is also the
Madlenerhaus (1986 m), which might be a club hut.

A long walk on the west side of the Silvretta Stausee (reservoir) follows,
leading to an even longer journey up the broad and peaceful Klostertal to the
Rote Furka and entry to Switzerland (it appears the walk might be on snow of the
Klostertaler Gletscher for a brief period). The Silvretta Hütte provides shelter
a ways down to the southwest, and then our route leaves the high peaks for a
subalpine valley walk to the west, along the Verstandclabach, to Novai, which
appears on the map as a settlement but is probably devoid of businesses (at 1363
meters). Now a walk up valley to the south past Stutzegg to the Berghaus
Vereina, thence Säss, and finally high snowy country of the Jöri Valley with two
turquoise lakes (the Jöriseen) and the terminus of this group at the Winterlücke
(pass, 2787 m). We descent steeply to a road at Wägerhus (2207 m) then travel on
trail a bit apart from the road to the Flüelapass (2383 m), which also marks the
entry into the Albula Alps. Two gorgeous lakes decorate the pass, the
Schottensee and the Schwarzsee, and we’ll walk by them in the valley for a time
before turning up to the southeast into the great scree basin below the Piz
Radönt, and making for the Fuorcla Radönt at 2788 meters across more or less
permanent snowfields. Traveling due south down from the pass, we’ll follow
spotty bits of trail south and southwest to the Fuorcla da Grialetsch (pass,
2537 m). A Swiss alpine club hut is here as well, the Chamanna da Grialetsch
CAS, nestled beside two small lakes.

From here we turn sharply west, descending into the Dürr valley but it appears
we may be able to travel cross-country due west to meet our next junction,
rather than descending all the way to Dürrboden (2007 m). This new trail goes up
valley, apparently towards the glacier-bedecked Chüealphorn to the Scalettapass
at 2606 meters. Passing through, we’ll descend due south to the Alp Funtauna at
2192 meters.

A relaxing walk slowly up the valley of the Funtauna eventually reaches the
Chamanna Kesch (2630 m), below the imposing and glacier-clad Piz Kesch (3417
m). We have two possibilities here, one is to climb the Porchabella Glacier to
the Porta d’Es-cha (3008 m), for a quick descent to the Chamanna d’Es-cha, and
then to the Albulapass. This is attractive, both for the views and the
efficiency, but the Porchabella Glacier is significantly crevassed, and the ease
of access to rocks of the Porta d’Es-cha is uncertain. It could stress the
limits of lightweight equipment (i.e., aluminum crampons for tennis shoes), and
the reasonable risk threshold for a solo traveler on a glacier. The alternative
is to descend all the way to Bergün (1367 m), and follow a combination of trail
and road up to the point near the Albulapass where our route continues. I would
suggest that it’s more likely that this option be chosen, unless strong
assurances regarding the state of the glacier can be found. What’s more, owing
to the unusual nature of your lightweight equipment and esoteric (though
comprehensive) background regarding alpine travel, it would be hard to find
someone with experience of the route over the Porta d’Es-cha to judge your
ability to overcome it. Thus far in the description of stages, this is the
riskiest proposition yet.


The Rote Furka (pass) above the Silvretta Hütte marks the entry into
Switzerland. It’s likely that the hut will accept Euros, however there isn’t a
real opportunity to acquire Swiss Francs until at least the Albula Pass, far to
the southwest. The route itself avoids towns and this could be an issue that
needs attention, perhaps by indicating detours that make town visits possible.


\stopcomponent
