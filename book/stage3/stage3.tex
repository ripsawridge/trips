\startcomponent stage3
\product quicksilver-book
\project quicksilver

\chapter{Stage Three}

Into Switzerland.

From the road, pass and houses at the Zeinisjoch we walk around either side of
the Stausee Kops lake (the left side is a dam), and climb gently up to a fine
pass view from the Breitspitze (2196 m), a suitable entry to the Silvretta
Alps. A descent to the valley floor of the Vallula to around 1700 meters
follows, then a climb to an eastern exit before reaching the upper valley to
gain a pass I call “P2515;” it resides just north of the Kleine Vallüla summit.
From here a long, gradually traversing descent to the Silvretta Haus at the
Biehlerhöhe (~2030 m) is accomplished on a meadowy southern alp called the
Maißboden. Possibly, Swiss Francs could be acquired here. There is also the
Madlenerhaus (1986 m), which might be a club hut.

A long walk on the west side of the Silvretta Stausee (reservoir) follows,
leading to an even longer journey up the broad and peaceful Klostertal to the
Rote Furka and entry to Switzerland (it appears the walk might be on snow of the
Klostertaler Gletscher for a brief period). The Silvretta Hütte provides shelter
a ways down to the southwest, and then our route leaves the high peaks for a
subalpine valley walk to the west, along the Verstandclabach, to Novai, which
appears on the map as a settlement but is probably devoid of businesses (at 1363
meters). Now a walk up valley to the south past Stutzegg to the Berghaus
Vereina, thence Säss, and finally high snowy country of the Jöri Valley with two
turquoise lakes (the Jöriseen) and the terminus of this group at the Winterlücke
(pass, 2787 m). We descent steeply to a road at Wägerhus (2207 m) then travel on
trail a bit apart from the road to the Flüelapass (2383 m), which also marks the
entry into the Albula Alps. Two gorgeous lakes decorate the pass, the
Schottensee and the Schwarzsee, and we’ll walk by them in the valley for a time
before turning up to the southeast into the great scree basin below the Piz
Radönt, and making for the Fuorcla Radönt at 2788 meters across more or less
permanent snowfields. Traveling due south down from the pass, we’ll follow
spotty bits of trail south and southwest to the Fuorcla da Grialetsch (pass,
2537 m). A Swiss alpine club hut is here as well, the Chamanna da Grialetsch
CAS, nestled beside two small lakes.

From here we turn sharply west, descending into the Dürr valley but it appears
we may be able to travel cross-country due west to meet our next junction,
rather than descending all the way to Dürrboden (2007 m). Here ends Stage Three.

\startitemize[1]
\item Alpengasthof Zeinisjoch (~1825 m), +43 5443 8233,
  http://www.zeinisjoch.com/. Private.
  Half-pension 48 Euros with private room.
\item Silvretta Haus (~2330 m), +43 5558 4246, http://www.silvretta-haus.at/.
  Private hotel, approximately 50 euros per night, but with an extensive restaurant, sauna and amenities.
\item Silvretta Hütte (2341 m), +41 81 422 13 06, http://silvrettahuette.ch/. SAC Hut. 65 sleeping places.
\item Berghaus Vereina (1944 m), +41 81 422 11 97,
  http://www.berghausvereina.ch/. Private hut.
  40 sleeping places. Bus transport to and from Klosters.
\item Passhotel Flüela Hospiz (2383 m), +41 81 416 17 47,
  http://www.flueela-hospiz.ch/. Private hut, but there is a common sleeping room.
\item Chamanna da Grialetsch (2542 m), +41 81 416 34 36,
  http://www.grialetsch.ch/. SAC Hut. 61 sleeping places. On the Kesch Trek and the Bünden Haute Route.
\stopitemize

\trfigure{}{The Silvretta Stausee.}{s3_1.pdf}
\trfigure{}{The Vereina Hütte.}{s3_2.pdf}
\trfigure{}{The Flüela Pass.}{s3_3.pdf}

\stopcomponent
